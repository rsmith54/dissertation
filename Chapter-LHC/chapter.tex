%This is the first chapter of the dissertation

%The following command starts your chapter. If you want different titles used in your ToC and at the top of the page throughout the chapter, you can specify those values here. Since Columbia doesn't want extra information in the headers and footers, the "Top of Page Title" value won't actually appear.

\chapter[The Large Hadron Collider][Top of Page Title]{The Large Hadron Collider}

This brief chapter will summarize the very basics of accelerator physics.
We will describe the CERN accelerator complex, with particular focus on the Large Hadron Collider (LHC).

\section{Basics of Accelerator Physics}

This section follows closely the presentation of \cite{ShiltsevColliderLectures}.

Simple particle accelerators simply rely on the acceleration of charged particles in a static electric field.
Given a field of strength $E$, charge $q$, and mass $m$, this is simply
\begin{equation}
a = \frac{qE}{m}.
\end{equation}
This was used for many early accelerators \todo{cite some?}
For a given particle with a given mass and charge, this is of course limited by the static electric field which can be produced.
This is limited by the electric breakdown at high voltages.

There are two complementary solutions to this issue.
First, we use the \textit{radio frequency acceleration} technique.
This consist of using a time-varied electric field.
We call the devices used for this \textit{RF cavities}.
Second, one bends the particles in a magnetic field, which allows them to pass through the same RF electric field over and over.
This second process is limited by \textit{synchrotron radiation}, which describes the radiation produced when a charged particle is accelerated.
The power radiated is
\begin{equation}
P \order \frac{1}{r^2} \begin{pmatrix} E/m \end{pmatrix}^4
\end{equation}
where $r$ is the radius of curvature and $E,m$ is the energy (mass) of the charged particle.
Given an energy which can be produced by a given set of RF cavities (which is \textit{not} limited by the mass of the particle), one then has two options to increase the actual collision energy : increase the radius of curvature or use a heavier particle.
Practically speaking, the easiest options for particles in a collider are protons and electrons, since they are (obviously) copious in nature and do not decay\footnotemark.
\footnotetext{Muon colliders are a really cool option at high energies, since the relativistic $\gamma$ factor gives them a relatively long lifetime in the lab frame.}
Given the dependence on mass, we can see why protons are used to reach the highest energies.
The tradeoff for this is that protons are not point particles, and we thus we don't know the exact incoming four-vectors of the protons, as discussed in Ch.\ref{ch:sm}.

The primary ``unit'' of a proton collider is the (proton) \textit{bunch}.
Bunches of protons are induced by the RF cavities; particles are accelerated or deccelerated by the cavities, and pushed together into bunches, which eventually pass through the RF cavities at the frequency of the cavity.
Besides the energy of the beam, the most important quantity to characterize a beam is known as the \textit{emittance}. \todo{add fig of emittance}
The emittance is a description of the size of the bunch ellipse.
The emittance is important mostly due to its influence on the \textit{instaneous luminosity}, which directly effects the rate of a given physics process.
For process of cross-section $\sigma$, the rate is given by
\begin{equation}
R = L \sigma
\end{equation}
where $L$ is the instaneous luminosity, given by:
\begin{equation}
L = \frac{f_{\text{rev}} N_b^2 R}{4 \pi \sigma^2}= \frac{f_{\text{rev}} n N_b^2 R }{4 \pi \beta^\star \beta_{\text{rel}} \epsilon}
\end{equation}

\section{Accelerator Complex}

\section{Large Hadron Collider}
