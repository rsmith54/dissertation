%This is the first chapter of the dissertation

%The following command starts your chapter. If you want different titles used in your ToC and at the top of the page throughout the chapter, you can specify those values here. Since Columbia doesn't want extra information in the headers and footers, the "Top of Page Title" value won't actually appear.

\chapter[The Standard Model][Top of Page Title]{The Standard Model}

Here you can write some introductory remarks about your chapter.
I like to give each sentence its own line.

When you need a new paragraph, just skip an extra line.

\section{Quantum Field Theory}

\todo{cite Yuval's lectures and notes somehow}

In this section, we provide a brief overview of the necessary concepts from Quantum Field Theory (QFT).

In modern physics, the laws of nature are described by the ``action'' $S$, with the imposition of the principle of minimum action. \todo{cite}
The action is the integral over the spacetime coordinates of the ``Lagrangian density'' \Lagr, or Lagrangian for short.
The Lagrangian is a function of ``fields''; general fields will be called $\phi(x^\mu)$, where the indices $\mu$ run over the space-time coordinates.
We can then write the action $S$ as

\begin{equation}
S = \int d^4 x \Lagr[ \phi_i(x^\mu) , \dmu \phi_i(x^\mu)]
\end{equation}

where we have an additional summation over $i$ (of the different fields).
Generally, we impose the following constraints on the Lagrangian :

\begin{enumerate}
\item Translational invariance - The Lagrangian is only a function of the fields $\phi$ and their derivatives $\dmu \phi$
\item Locality - The Lagrangian is only a function of one point $x_\mu$ in spacetime.
\item Reality condition - The Lagrangian is real to conserve probability.
\item Lorentz invariance - The Lagrangian is invariant under the \Poincare group of spacetime.
\item Analyticity - The Lagrangian is an analytical function of the fields; this is to allow the use of pertubation theory.
\item Invariance and Naturalness - The Lagrangian is invariant under some internal symmetry groups; in fact, the Lagrangian will have \textit{all} terms allowed by the imposed symmetry groups. \todo{maybe add in ref here}
\item Renormalizabilty - The Lagrangian will be renormalizable - in practice, this means there will not be terms with more than power 4 in the fields.
\end{enumerate}

The key item from the point of view of this thesis is that of ``Invariance and Natural''.
We impose a set of ``symmetries'' and then our Lagragian is the most general which is allowed by those symmetries.

\section{Symmetries}

Symmetries can be seen as the fundamental guiding concept of modern physics.
Symmetries are described by ``groups''. \todo{cite?}.
To illustrate the importance of symmetries and their mathematical description, groups, we start here with two of the simplest and most useful examples :  \Ztwo and $U(1)$.

\subsection{\Ztwo symmetry}

\Ztwo symmetry is the simplest example of a ``discrete'' symmetry.
Consider the most general Lagrangian of a single real scalar field $\phi(x_\mu)$

\begin{equation} \label{scalarFieldLagrangian}
\Lagr_\phi = \frac{1}{2} \dmu \phi \dmuup \phi - \frac{m^2}{2} \phi^2 - \frac{\mu}{2 \sqrt{2}}  \phi^3 - \lambda \phi^4
\end{equation}

Now we \textit{impose} the symmetry
\begin{equation}
\Lagr(\phi) = \Lagr(- \phi)
\end{equation}

This has the effect of restricting the allowed terms of the Lagrangian.
In particular, we can see the term $\phi^3 \rightarrow - \phi^3$ under the symmetry transformation, and thus must be disallowed by this symmetry.
This means under the imposition of this particular symmetry, our Lagrangian should be rewritten as

\begin{equation}
\Lagr_\phi = \frac{1}{2} \dmu \phi \dmuup \phi - \frac{m^2}{2} \phi^2  - \lambda \phi^4
\end{equation}

The effect of this symmetry is that the total number of  $\phi$ particles can only change by even numbers, since the only interaction term $\lambda \phi^4$ is an even power of the field.
This symmetry is often imposed in supersymmetric theories, as we will see in Chapter 3.

\subsection{$U(1)$ symmetry}

$U(1)$ is the simplest example of a continuous (or \textit{Lie}) group.
Now consider a theory with a single complex scalar field $\phi = \operatorname{Re}\phi + i \operatorname{Im}\phi$

\begin{equation}
\Lagr_\phi = \delta_{i,j} \frac{1}{2} \dmu \phi_i \dmuup \phi_j - \frac{m^2}{2} \phi_i \phi_j - \frac{\mu}{2 \sqrt{2}}  \phi_i \phi_j \phi_k  - \lambda \phi_i \phi_j \phi_k \phi_l
\end{equation}

where $i,j,k,l = Re, Im$.
In this case, we impose the following $U(1)$ symmetry : $\phi \rightarrow e^{i\theta}, \phi^* \rightarrow e^{-i\theta} $.
We see immediately that this again disallows the third-order terms, and we can write a theory of a complex scalar field with $U(1)$ symmetry as

\begin{equation}
\Lagr_\phi =  \dmu \phi \dmuup \phi^* - \frac{m^2}{2} \phi \phi^* -   - \lambda (\phi \phi^*)^2
\end{equation}

\section{Local symmetries}

The two examples considered above are ``global'' symmetries in the sense that the symmetry transformation does not depends on the spacetime coordinate $x_\mu$.
We know look at local symmetries; in this case, for example with a local $U(1)$ symmetry,  the transformation has the form $\phi(x_\mu) \rightarrow e^{i \theta (x_mu)} \phi(x_\mu)$.
These symmetries are also known as ``gauge'' symmetries; all symmetries of the Standard Model are gauge symmetries.

There are wide-ranging consequences to the imposition of local symmetries.
To begin, we note that the derivative terms of the Lagrangian \ref{scalarFieldLagrangian} are \textit{not} invariant under a local symmetry transformation
\begin{equation}
\dmu \phi(x_\mu) \rightarrow \dmu ( e^(i\theta(x_\mu) \phi(x_\mu )) = (1 + i \theta(x_\mu) ) e^(i\theta(x_\mu) \phi(x_\mu )
\end{equation}\todo {GET THIS RIGHT}

This leads us to note that the kinetic terms of the Lagrangian are also not invariant under a gauge symmetry.
This would lead to a model with no dynamics, which is clearly unsatisfactory.

Let us take inspiration from the case of global symmetries.
We need to define a so-called ``covariant'' derivative $\Dmuup$ such that

\begin{equation}
\begin{aligned}
\Dmuup \phi   \rightarrow e^{ i q \theta(x^\mu) \Dmuup \phi} \\
\Dmuup \phi^* \rightarrow e^{-i q \theta(x^\mu) \Dmuup \phi} \\
\end{aligned}
\end{equation}

Since $\phi$ and$\phi^*$ transforms with the opposite phase, this will lead the invariance of the Lagrangian under our local gauge transformation.
This $\Dmuup$ is of the following form

\begin{equation}
\Dmuup = \dmu - i g q A^\mu
\end{equation}

where $A^\mu$ is a vector field we introduce with the transformation law

\begin{equation}
A^\mu \rightarrow A^\mu - \frac{1}{g} \dmu \theta
\end{equation}

and $g $ is the coupling constant associated to vector field.
This vector field $A^\mu$ is also known as a ``gauge'' field.

Since we need to add all allowed terms to the Lagrangian, we define

\begin{equation}
F^{\mu\nu} = A^\mu A^\nu - A^\nu A^\mu
\end{equation}

and then we must also add the kinetic term :

\begin{equation}
\Lagr_{\text{gauge}} = - \frac{1}{4} F^{\mu\nu} F_{\mu\nu}
\end{equation}

The most general renormalizable Lagrangian with fermion and scalar fields can be written in the following form

\begin{equation}
\Lagr = \Lagr_{kin} + \Lagr_{\phi} + \Lagr_\psi +   \Lagr{Yukawa}
\end{equation}

\subsection{Symmetry breaking and the Higgs mechanism}
\label{subsec:symmetry_breaking}
Here we view some examples of symmetry breaking.
We investigate breaking of a global $U(1)$ symmetry and a local $U(1)$ symmetry.
The SM will break the electroweak symmetry $SU(2) x U(1)$, and in Chapter 3 we will see how supersymmetry must also be broken.

There are two ideas of symmetry breaking
\begin{itemize}
\item Explicit symmetry breaking by a small parameter - in this case, we have a small parameter which breaks an ``approximate'' symmetry of our Lagrangian.
An example would be the theory of the single scalar field \ref{scalarFieldLagrangian}, when $\mu << m^2$ and $\mu << \lambda$.
In this case, we can often ignore the small term when considering low-energy processes.
\item Spontaneous symmetry breaking (SSB) - spontaneous symmetry breaking occurs when the Lagrangian is symmetric with respect to a given symmetry transformation, but the ground state of the theory is \textit{not} symmetric with respect to that transformation.
This can have some fascintating consequences, as we will see in the following examples
\end{itemize}
Symmetry breaking a

\subsubsection{U(1) global symmetry breaking}

Consider the theory of a complex scalar field under the $U(1)$ symmetry, or the transformation
\begin{equation}
\phi \rightarrow e^{i\theta} \phi
\end{equation}

The Lagrangian for this theory is
\begin{equation}
\Lagr = \dmuup \phi^{\dag} \dmu \phi + \frac{\mu^2}{2} \phi^{\dag} \phi + \frac{\lambda}{4} (\phi^\dag \phi)^2
\end{equation}

Let us write this theory in terms of two scalar fields, $h$ and $\xi$ : $\phi = (h + i\xi) / \sqrt(2)$.
The Lagrangian can then be written as
\begin{equation}
\Lagr = \dmuup h \dmu h + \dmuup \xi dmu \xi - \frac{\mu^2}{2} (h^2 + \xi^2) - \frac{\lambda}{4}(h^2 + \xi^2)^2
\end{equation}

First, note that the theory is only stable when $\lambda > 0$.
To understand the effect of SSB, we now enforce that $\mu^2 < 0$, and define $v^2 = -\mu^2/\lambda$.
We can then write the scalar potential of this theory as :
\begin{equation}
V(\phi) = \lambda (\phi^\dag \phi - v^2/2)^2
\end{equation}

Minimizing this equation with respect to $\phi$, we can see that the ``vacuum expectation value'' of the theory is
\begin{equation}
2<\phi^\dag \phi> = <h^2 + \xi^2 > = v^2
\end{equation}

We now reach the ``breaking'' point of this procedure.
In the $(h, \xi)$ plane, the minima form a circle of radius $v$.
We are free to choose any of these minima to expand our Lagrangian around; the physics is not affected by this choice.
For convenience, choose $<h> = v, <\xi^2> = 0$.

Now, let us define $h' = h - v , \xi' = \xi $ with VEVs $<h'> = 0 , <\xi'> = 0$.
We can then write our spontaneously broken Lagrangian in the form
\begin{equation}
\Lagr = \frac{1}{2} \dmu h' \dmuup h' +  \frac{1}{2} \dmu \xi' \dmuup \xi' - \lambda v^2 h'^2 - \lambda v h' (h'^2 + \xi'^2 ) - \lambda (h'^2 + \xi'^2)^2
\end{equation}


\todo{CITE THIS PICTURE}
\begin{figure} \label{fig:sombrero}
\caption{Sombrero potential}
\includegraphics[width=\linewidth]{sombrero_potential}
\end{figure}

\subsubsection{U(1) local  symmetry breaking}

\section{The Standard Model}

\subsection{Overview}

The Standard Model is another name for the theory of the internal symmetry group $SU(3)_C \otimes SU(2)_L \otimes U(1)_Y$ \todo{CHECK}.
This quantum field theory is the culmination of years of work in both theoretical and particle physics.  \todo{cite}

\todo{CITE THIS PICTURE}
\begin{figure}
\caption{The interactions of the Standard Model}
\includegraphics[width=\linewidth]{Standard_Model_Feynman_Diagram_Vertices}
\end{figure}

\subsection{Field Content}

The SM field content is
\begin{equation}
\begin{aligned}
\text{Fermions } Q_L(3,2)_{+1/3}, \xspace  U_R(3,1)_{+4/3},\xspace  D_R(3,1)_{-2/3} ,\xspace  L_L(1,2)_{-1} ,\xspace  E_R(1,1)_{-2}\\
\text{Scalar (Higgs) } \xspace \phi(1,2)_{+1} \\
\text{Vector Fields } \xspace G^\mu(8,1)_0 \xspace W^\mu(1,3)_0  \xspace B^\mu(1,1)_0
\end{aligned}
\end{equation}
where the $(A, B)_Y$ notation represents the irreducible representation under $SU(3)$ and $SU(2)$, with $Y$ being the electroweak hypercharge.
Each of these fields has an additional index, representing the three generation of fermions.

We observed that $Q_L, U_R,$ and $D_R$ are triplets under $SU(3)_C$; these are the \textit{quark} fields.
The ``color'' group, $SU(3)_C$ is mediated by the ``gluon'' field $G^\mu(8,1)_0$, which has 8 degrees of freedom; we say there are 8 gluons.
The fermion fields $L_L(1,2)_{-1}$ and $  E_R(1,1)_{-2} $ are singlets under $SU(3)_C$; we call them \textit{leptons}.

Next, we note the ``left-handed'' (``right-handed'') fermion fields, denoted by $L$ ($R$) subscript,
The left-handed fields form doublets under $SU(2)_L$.
These are mediated by the three degrees of freedom of the  ``W'' fields $W^\mu(1,3)_0$.
These fields only act on the left-handed particles of the Standard Model.
This is the reflection of the ``chirality'' of the Standard Model; the left-handed and right-handed particles are treated differently by the electroweak forces.
The right-handed fields, $U_R, D_R$, and $E_R$, are singlets under $SU(2)_L$.

The $U(1)_Y$ symmetry is associated to the $B^\mu(1,1)_0$ boson with one degree of freedom.
The charge $Y$ is known as the electroweak hypercharge.

\subsection{$\Lagr_{kin}$}

For each of the vector boson fields, we have the follow field strengths :

\begin{equation}
\begin{aligned}
G^{\mu\nu}_a = \dmuup G^\nu_a + \partial^\nu G^\mu_a - g_s f_{abc} G^\mu_b G^\nu_c \\
W^{\mu\nu}_a = \dmuup W^\nu_a + \partial^\nu W^\mu_a - g \epsilon_{abc} W_b^\mu W_c^\nu \\
B^{\mu\nu}   = \dmuup B^\nu   + \partial^\nu B^\mu
\end{aligned}
\end{equation}

where $g$ and $g_s$ are the electroweak and strong coupling constant.

We can write the covariant derivative for the Standard Model as
\begin{equation}
\Dmuup = \dmuup + ig_s G^\mu_a L_a + i g W^\mu_a T_a + i g' Y B^\mu
\end{equation}
where $L_a$ and $T_a$ are the generators of $SU(3)_C $ and $SU(2)_L$ respectively for each of the representations.
Explicitly, for the $SU(3)_C$ triplets, $L_a = \frac{1}{2} \lambda_a$ and for the $SU(3)_C$ singlets, $L_a = 0$. \todo{GELLMANN and Pauli matrices}.
For $SU(2)_L$ doublets, $T_a = \frac{1}{2} \sigma_a $ and for $SU(2)_L$ singlets, $T_a = 0$.

The combination of these terms allows us to write the kinetic terms of the Lagrangian as
\begin{equation}
\begin{aligned}
\Lagr_{kin} = G^{\mu\nu} G_{\mu\nu} + W^{\mu\nu} W_{\mu\nu} + B^{\mu\nu} B_{\mu\nu}\\
 + \Dmuup Q_L \Dmu Q_L + \Dmuup U_R \Dmu U_R +  \Dmuup D_R \Dmu D_R + \Dmuup L_L \Dmu L_LL + \Dmuup E_R \Dmu E_R
\end{aligned}
\end{equation}

\subsection{$\Lagr_{\psi}$ }

We cannot write down any mass terms for fermions in the Standard Model.
Dirac mass terms are forbidden since they are all assigned to ``chiral'' representations of the gauge symmetry.
Majorana mass terms are disallowed since there are no fields with $Y \slashed{=} 0$.

\subsection{$\Lagr_{Yuk}$ }

We write the Yukawa portion of the Standard Model Lagrangian

\begin{equation}
\Lagr_{Yuk} = Y_{ij}\bar{L_{Li} E_{Rj}} \phi + h.c.
\end{equation}

The Yukawa matrix $Y$ is a general complex 3 $\times$ 3 matrix of dimensionless couplings which can be diagonalized, leading to a diagonal matrix with only three real parameters $(y_e , y_\mu , y_\tau)$.
This reflects the fact that for the electron, muon, and tau lepton, the interaction basis is the same as the mass basis; this is the same as saying an electron has a well-defined mass.

\section{$\Lagr_\phi$, Electroweak Symmetry breaking and the Higgs Boson}

Let us now recall that local gauge invariance means that the vector fields in this theory are \textit{massless}.
N\"aively, it seems this combined with the chirality of the Standard Model, that \textit{none} of the fields have masses.
The solution to this seeming conundrum is of course the well-known ``Higgs'' mechanism, described in Sec. \ref{subsec:symmetry_breaking}.

In the Standard Model, the Higgs potential is given by
\begin{equation} \label{eq:higgs_potential}
\Lagr_\phi = -\mu^2 \phi^\dagger \phi - \lambda (\phi^\dagger \phi)^2.
\end{equation}

Since $\lambda$ is dimensionless and real, to have a potential bounded from below, we require $\lambda > 0$.
To break the gauge symmetry, we require $\mu^2 < 0$, leading again to the sombrero potential \ref{fig:sombrero}.
We define
\begin{equation}
v^2 = - \frac{\mu^2}{\lambda}.
\end{equation}

This allows us to write \ref{eq:higgs_potential} as
\begin{equation} \label{eq:higgs_potential_rewritten}
\Lagr_\phi = - \lambda (\phi^\dagger \phi - \frac{v^2}{2})^2
\end{equation}
after dropping the constant term.

This means the $\phi$ field acquires a VEV $|<\phi>| = v/\sqrt{2}$.
Choosing the convenient gauge
\begin{equation}
\phi = \begin{pmatrix} 0 \\ v/\sqrt{2} \end{pmatrix},
\end{equation}

The VEV breaks the $SU(2)_L \otimes U(1)_Y$ symmetry to a $U(1)_{EM}$ subgroup.
We can identify the unbroken generator of this $U(1)_{EM}$ subgroup as $Q_{EM} = T_3 + Y/2$, since this vanishes in the down component
\begin{equation}
Q_{\gamma} \phi = (T_3 + Y/2) \phi = (\frac{1}{2} \sigmathree + \frac{1}{2} I ) \begin{pmatrix} 0 \\ v/\sqrt{2} \end{pmatrix}.
\end{equation}
Here we see the indicative $\gamma$ for the photon, as this unbroken $U(1)_{EM}$ symmetry is of course the symmetry associated to the electromagnetic force mediated by the gauge boson known as the photon.

There are three broken generators : $T_1, T_2, T_3 - Y/2$.
These are each associated to one of the massive gauge bosons induced by the symmetry breaking.
Choosing a gauge which rotates away the ``eaten'' Goldstone boson degrees of freedom, we can write the Higgs field as
\begin{equation}
\label{eq:higgs_field}
\phi = \frac{1}{\sqrt{2}}\begin{pmatrix} 0 \\ v + h(x) \end{pmatrix}.
\end{equation}

\section{Particle Spectrum : Standard Model Lagrangian after Electroweak Symmetry Breaking}

We can now return to the Standard Model Lagrangian and use the equation for the Higgs field after EWSB \ref{eq:higgs_field}.
This will show us the ``physical'' particle content of the Standard Model.

\subsection{Particle content associated to $\Lagr_\phi$}

Setting $phi$ as in Eq.\ref{eq:higgs_field}, we quickly see that we can rewrite Eq.\ref{eq:higgs_potential_rewritten} as
\todo{ CHECK FACTORS OF TWO}
\begin{equation}
\Lagr_\phi = - \lambda (\phi^\dagger \phi - \frac{v^2}{2})^2  = - \lambda ( \frac{1}{2} (v + h(x))^2 - \frac{v^2}{2})^2 = - \lambda ( h(x)^2 + vh(x))^2 = -\lambda ( h(x)^4 + v h(x)^3 + \frac{v^2}{2} h(x)^2 ).
\end{equation}

Interpreting the Higgs field squared term as the mass term of the Higgs boson, we see that $m_H = \sqrt{2 \lambda} v$.

\subsection{Particle content associated to $\Lagr_{kin}$}

Again using Eq.\ref{eq:higgs_field} and $\Dmuup = \dmuup + ig_s G^\mu_a L_a + i g W^\mu_a T_a + i g' Y B^\mu $, we can see how the mass terms associated to the three massive gauge bosons, and also see how the photon stays massless.
The mass terms for the gauge boson fields come from the kinetic term of the Higgs field :
\begin{equation}
\begin{aligned}
\Lagr_{M_V} = \Dmuup \phi \Dmu \phi = (i g W^\mu_a T_a + i g' Y B^\mu ) \frac{1}{\sqrt{2}}\begin{pmatrix} 0 \\ v \end{pmatrix} (i g W_{\mu,a} T_a + i g' Y B_\mu ) \frac{1}{\sqrt{2}}\begin{pmatrix} 0 \\ v \end{pmatrix} = \\
\frac{1}{8} |\begin{pmatrix} gW_3 + g'B & g(W_1 - iW_2) \\ g(W_1 + iW_2) & -gW_3 + g'B \end{pmatrix}  \begin{pmatrix} 0 \\ v \end{pmatrix} |^2
\end{aligned}
\end{equation}
where we have noted that $\dmu$ and $L_a$ both disappear when acting on $\phi$.
Defining the \textit{Weinberg} angle $\tan(\theta_W) = g'/g$ and the following physical fields :
\begin{equation}
\begin{aligned}
W^{\pm} = \frac{1}{\sqrt{2}}(W_1 \mp iW_2) \\
Z^0 = \cos \theta_W W_3 - \sin\theta_W B \\
A^0 = \sin \theta_W W_3 + \cos\theta_W B
\end{aligned}
\end{equation}
we see that we can write the piece of the Lagrangian associated to the vector boson masses as
\begin{equation}
\Lagr_{M_V} = \frac{1}{4} g^2 v^2 W^+ W^- + \frac{1}{8} (g^2 + g'^2)v^2 Z^0 Z^0 .
\end{equation}
and we have the following values of the masses for the vector bosons :
\begin{equation}
\begin{aligned}
m_W^2 = \frac{1}{4} g^2 v^2 \\
m_Z^2 = \frac{1}{4} (g^2 + g'^2) v^2 \\
m_A^2 = 0
\end{aligned}
\end{equation}

\section{Deficiencies of the Standard Model}

By using the asterisk to start a new section, I keep the section from appearing in the table of contents.
If you want your sections to be numbered and to appear in the table of contents, remove the asterisk.
