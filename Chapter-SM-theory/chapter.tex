%This is the first chapter of the dissertation

%The following command starts your chapter. If you want different titles used in your ToC and at the top of the page throughout the chapter, you can specify those values here. Since Columbia doesn't want extra information in the headers and footers, the "Top of Page Title" value won't actually appear.

\chapter[The Standard Model][Top of Page Title]{The Standard Model}

Here you can write some introductory remarks about your chapter.
I like to give each sentence its own line.

When you need a new paragraph, just skip an extra line.

\section{Quantum Field Theory}

\todo{cite Yuval's lectures and notes somehow}

In this section, we provide a brief overview of the necessary concepts from Quantum Field Theory (QFT).

In modern physics, the laws of nature are described by the ``action'' $S$, with the imposition of the principle of minimum action. \todo{cite}
The action is the integral over the spacetime coordinates of the ``Lagrangian density'' \Lagr, or Lagrangian for short.
The Lagrangian is a function of ``fields''; general fields will be called $\phi(x^\mu)$, where the indices $\mu$ run over the space-time coordinates.
We can then write the action $S$ as

\begin{equation}
S = \int d^4 x \Lagr[ \phi_i(x^\mu) , \dmu \phi_i(x^\mu)]
\end{equation}

where we have an additional summation over $i$ (of the different fields).
Generally, we impose the following constraints on the Lagrangian :

\begin{enumerate}
\item Translational invariance - The Lagrangian is only a function of the fields $\phi$ and their derivatives $\dmu \phi$
\item Locality - The Lagrangian is only a function of one point $x_\mu$ in spacetime.
\item Reality condition - The Lagrangian is real to conserve probability.
\item Lorentz invariance - The Lagrangian is invariant under the \Poincare group of spacetime.
\item Analyticity - The Lagrangian is an analytical function of the fields; this is to allow the use of pertubation theory.
\item Invariance and Naturalness - The Lagrangian is invariant under some internal symmetry groups; in fact, the Lagrangian will have \textit{all} terms allowed by the imposed symmetry groups. \todo{maybe add in ref here}
\item Renormalizabilty - The Lagrangian will be renormalizable - in practice, this means there will not be terms with more than power 4 in the fields.
\end{enumerate}

The key item from the point of view of this thesis is that of ``Invariance and Natural''.
We impose a set of ``symmetries'' and then our Lagragian is the most general which is allowed by those symmetries.

\section{Symmetries}

Symmetries can be seen as the fundamental guiding concept of modern physics.
Symmetries are described by ``groups''. \todo{cite?}.
To illustrate the importance of symmetries and their mathematical description, groups, we start here with two of the simplest and most useful examples :  \Ztwo and $U(1)$.

\subsection{\Ztwo symmetry}

\Ztwo symmetry is the simplest example of a ``discrete'' symmetry.
Consider the most general Lagrangian of a single real scalar field $\phi(x_\mu)$ :

\begin{equation} \label{scalarFieldLagrangian}
\Lagr_\phi = \frac{1}{2} \dmu \phi \dmuup \phi - \frac{m^2}{2} \phi^2 - \frac{\mu}{2 \sqrt{2}}  \phi^3 - \lambda \phi^4
\end{equation}

Now we \textit{impose} the symmetry :
\begin{equation}
\Lagr(\phi) = \Lagr(- \phi)
\end{equation}

This has the effect of restricting the allowed terms of the Lagrangian.
In particular, we can see the term $\phi^3 \rightarrow - \phi^3$ under the symmetry transformation, and thus must be disallowed by this symmetry.
This means under the imposition of this particular symmetry, our Lagrangian should be rewritten as :

\begin{equation}
\Lagr_\phi = \frac{1}{2} \dmu \phi \dmuup \phi - \frac{m^2}{2} \phi^2  - \lambda \phi^4
\end{equation}

The effect of this symmetry is that the total number of  $\phi$ particles can only change by even numbers, since the only interaction term $\lambda \phi^4$ is an even power of the field.
This symmetry is often imposed in supersymmetric theories, as we will see in Chapter 3.

\subsection{$U(1)$ symmetry}

$U(1)$ is the simplest example of a ``continuous'' (or Lie) group.
Now consider a theory with a single complex scalar field $\phi = \operatorname{Re}\phi + i \operatorname{Im}\phi$ :

\begin{equation}
\Lagr_\phi = \delta_{i,j} \frac{1}{2} \dmu \phi_i \dmuup \phi_j - \frac{m^2}{2} \phi_i \phi_j - \frac{\mu}{2 \sqrt{2}}  \phi_i \phi_j \phi_k  - \lambda \phi_i \phi_j \phi_k \phi_l
\end{equation}

where $i,j,k,l = Re, Im$.
In this case, we impose the following $U(1)$ symmetry : $\phi \rightarrow e^{i\theta}, \phi^* \rightarrow e^{-i\theta} $.
We see immediately that this again disallows the third-order terms, and we can write a theory of a complex scalar field with $U(1)$ symmetry as :

\begin{equation}
\Lagr_\phi =  \dmu \phi \dmuup \phi^* - \frac{m^2}{2} \phi \phi^* -   - \lambda (\phi \phi^*)^2
\end{equation}

\section{Local symmetries}

The two examples considered above are ``global'' symmetries in the sense that the symmetry transformation does not depends on the spacetime coordinate $x_\mu$.
We know look at local symmetries; in this case, for example with a local $U(1)$ symmetry,  the transformation has the form $\phi(x_\mu) \rightarrow e^{i \theta(\x_mu)}\phi(x_\mu)$.
These symmetries are also known as ``gauge'' symmetries; all symmetries of the Standard Model are gauge symmetries.

There are wide-ranging consequences to the imposition of local symmetries.
To begin, we note that the derivative terms of the Lagrangian \ref{scalarFieldLagrangian} are \textit{not} invariant under a local symmetry transformation :
\begin{equation}
\dmu \phi(x_\mu) \rightarrow \dmu ( e^(i\theta(x_\mu) \phi(x_\mu )) = (1 + i \theta(x_\mu) ) e^(i\theta(x_\mu) \phi(x_\mu )
\end{equation}\todo {GET THIS RIGHT}

\section{The Standard Model}

\subsection{Overview}

By using the asterisk to start a new section, I keep the section from appearing in the table of contents.
If you want your sections to be numbered and to appear in the table of contents, remove the asterisk.


\subsection{Fermions}

By using the asterisk to start a new section, I keep the section from appearing in the table of contents.
If you want your sections to be numbered and to appear in the table of contents, remove the asterisk.

\subsection{Bosons}

By using the asterisk to start a new section, I keep the section from appearing in the table of contents.
If you want your sections to be numbered and to appear in the table of contents, remove the asterisk.

\section{Electroweak Symmetry breaking and the Higgs Boson}

By using the asterisk to start a new section, I keep the section from appearing in the table of contents.
If you want your sections to be numbered and to appear in the table of contents, remove the asterisk.

\section{Deficiencies of the Standard Model}

By using the asterisk to start a new section, I keep the section from appearing in the table of contents.
If you want your sections to be numbered and to appear in the table of contents, remove the asterisk.
