%This is the first chapter of the dissertation

%The following command starts your chapter. If you want different titles used in your ToC and at the top of the page throughout the chapter, you can specify those values here. Since Columbia doesn't want extra information in the headers and footers, the "Top of Page Title" value won't actually appear.

\chapter[Supersymmetry][Top of Page Title]{Supersymmetry}\label{ch:susy}

This chapter will introduce supersymmetry (SUSY) \cite{Lykken:1996xt, susyPrimer} \todo{cite more}.\todo{cite SUSY lectures from presusy}.
We will begin by introducing the concept of a \textit{superspace}, and discuss some general ingredients of supersymmetric theories.
This will include a discussion of how the problems with the Standard Model described in Ch.\ref{ch:sm} are naturally fixed by these theories.

The next step is to discuss the particle content of the \textit{Minimally Supersymmetric Standard Model} (MSSM).
As its name implies, this theory contains the minimal additional particle content to make Standard Model supersymmetric.
We then discuss the important phenomonological consequences of this theory, especially as would be observed in experiments at the LHC.

\section{Supersymmetric theories : from space to superspace}

\subsection{Coleman-Mandula ``no-go'' theorm}

We begin the theoretical motivation for supersymmetry by citing the ``no-go'' theorem of Coleman and Mandula \cite{Coleman:1967ad}.
This theorem forbids \textit{spin-charge unification}; it states that all quantum field theories which contain nontrivial interactions must be a direct product of the \Poincare group of Lorentz symmetries, the internal product from of gauge symmetries, and the discrete symmetries of parity, charge conjugatio, and time reversal.
The assumptions which go into building the Coleman-Mandula theorem are quite restrictive, but there is one unique way out, which has become known as \textit{supersymmetry} \cite{Golfand:1971iw, Haag:1974qh}.



\todo{cite SUSY lectures from presusy}

A \textit{supersymmetric} transformation $Q$ transforms a bosonic state into a fermionic state, and vice versa :
\begin{aligned}
Q \braket{\text{Fermion}} &= \braket{\text{Boson}}
Q \braket{\text{Boson}} &= \braket{\text{Fermion}}
\end{aligned}
To ensure this relation holds, $Q$ must be an anticommuting spinor.
Additionally, since spinors are inherently complex, $Q^\dagger$ must also be a generator of the supersymmetry transformation.

\subsection{Only Additional allowed Lorentz invariant symmetry}
\subsection{Dark Matter}
\subsection{Cancellation of quadratic divergences in corrections to the Higgs Mass}

\section{Supersymmetry}

\section{Additional particle content}

\section{Phenomenology}

R parity
Consequences for sq/gl decays
