%This is the conclusion of the dissertation

\chapter[Conclusion][Conclusion]{Conclusion} %Your conclusion isn't a numbered chapter, so we use the asterisk here.

This thesis presented a search for supersymmetry in hadronic final states.
The dataset was near the highest integrated luminosity sample and used the highest $\sqrt{s}$ proton-proton collisions ever produced in a laboratory.
By the time this thesis is defended, the dataset will expanded further, and

The search detailed in this thesis is the first to use Recursive Jigsaw Reconstruction.
The analysis failed to find an excess, and strong limits were produced on the simplified models of sparticle pair production considered.
It is useful to consider what has been learned by both this analysis, and dozens of other searches for new physics at both ATLAS and CMS.

There are a few stray thoughts we would like to discuss here: $R$-parity conservation and the use of simplified models.
These disparate concepts can be synthesized into a

\todo{do I want to be this strong}
The assumption of $R$-parity is at the heart of a large number of LHC SUSY searches.
$R$-parity can not be too badly broken, as the proton is very stable, as discussed in the Introduction.
However, there is not a particularly good reason to assume that all the $R$-parity violating couplings are zero, as any individual one can be turned on while not inducing the proton decay shown in \Cref{fig:proton_decay}.
The problem is that the imposition of $R$-parity solves two other problems.
As discussed in the Introduction, $R$-parity conservation would lead to a dark matter candidate.
However, we consider this to be a bit of backwards logic which just happens to fit conveniently into another mystery.
Related, and more insidiously, the assumption of $R$-parity makes searches for SUSY much simpler, as \met can be used as a very strong discriminator against QCD backgrounds.
In order to probe the phase space of $R$-parity violating supersymmetry, much more robust ways of modeling QCD backgrounds is a must.

Simplified models provide a useful tool to understand the reach of supersymmetric searches. \todo{cite Martin}
However, they can also lead us astray, as we must make ad-hoc assumptions which are not well-motivated.
The assumption of $R$-parity is one example of this, but others exist.
Although they are not covered directly in this thesis, searches for supersymmetric tops are particularly affected by the branching ratio assumptions.
As both stops and tops have a variety of decay modes, the assumptions can drastically affect the final limits.
In future searches, there must be additional focus on understanding the simplified models inside of the larger space of the MSSM, or even some more complicated supersymmetric theories.

The space of supersymmetric models is \textit{very} large.
Even in the MSSM, we have 120 free parameters, which we have barely begun to explore.
Viewing the landscape from before Run-1, it is easy to see why the strategies detailed here became commonplace.
Essentially, we \textit{expected} to find some sort of new physics, which would help explain the hierarchy problem.
If we even discover one sparticle, with its associated mass and branching ratios, we would drastically reduce the number of free SUSY model parameters.

From our current point of view, this seems na{\"i}ve.
We have yet to find any supersymmetric particle, and much of the MSSM parameter space has been ruled out.
However, we should not yet despair, as there is much more phase space to be explored.
Things will just have to be a little more fun.

\todo{curse of dimensionality}