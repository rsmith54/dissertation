%This is the conclusion of the dissertation

\chapter[Conclusion][Conclusion]{Conclusion} %Your conclusion isn't a numbered chapter, so we use the asterisk here.

This thesis presented a search for supersymmetry in hadronic final states.
The dataset had near the highest integrated luminosity to date, and the proton-proton collisions had the highest center-of-mass energy every produced in a laboratory.

The search described in this thesis is the first to use Recursive Jigsaw Reconstruction.
RJR shows promise as the  conceptual successor to the razor technique.
It compares favorably with previous analysis strategies.
As no excess is observed, we set model-dependent and model-independent limits in models of sparticle pair production.
We consider more broadly what has been learned by this analysis and dozens of other null searches for new physics at both ATLAS and CMS.

The assumption of $R$-parity is at the heart of a large number of LHC SUSY searches.
$R$-parity can not be too badly broken, due to the stability of the proton, as discussed in \Cref{ch:intro,ch:susy}.
However, there is no good reason to assume that all the $R$-parity violating (RPV) couplings are zero.
Any individual RPV coupling can be nonzero, while still avoiding the proton decay shown in \Cref{fig:proton_decay}.
The imposition of $R$-parity has two significant other effects.

$R$-parity conservation leads to a dark matter candidate.
Indeed, this candidate can be a WIMP, and this lucky coincidence is often known as the ``WIMP miracle'' \cite{darkMatterPrimer}.
However, it is possible that this miracle is simply a red herring.
The dark matter could be of a different nature than a weakly interacting massive particle, even assuming we discover supersymmetry with an appropriate LSP.
Additionally, the WIMPS could be real, but not coincide with the LSP from supersymmetry.
As evidence for dark matter is the best experimental motivation for supersymmetry, contemplation of these scenarios does not inspire confidence.

$R$-parity conservation makes searches for supersymmetry significantly easier.
In SUSY searches where $R$-parity is conserved, \met or related variables are strong discriminators against the dominant QCD background.
If $R$-parity is violated, the LSP will decay via SM particles, which can be measured by our experiments.
RPV searches do not have these discriminators against the most complicated background.
In order to more completely cover the phase space of $R$-parity violating supersymmetry, much more robust techniques to understand QCD backgrounds will be needed.

Simplified models provide a useful tool to understand the reach of supersymmetric searches~\cite{whitePresusy}.
However, they can also lead us astray, as we make ad-hoc assumptions.
Although not covered directly in this thesis, searches for supersymmetric tops are particularly affected by branching ratio assumptions.
As both stops and tops have a variety of decay modes, assumptions can drastically affect the final limits.
In future searches, it is imperative to understand simplified models inside of the larger space of the MSSM and more complicated supersymmetric models.

The space of supersymmetric models is \textit{very} large.
Even in the MSSM, we have 120 free parameters.
The total space of the MSSM is very large.
Viewing the landscape from before Run-1, it is easy to see why the strategies of ATLAS and CMS became commonplace.
We expected to find some sort of new physics, which would help explain the hierarchy problem.
If we even discover one sparticle, with its associated mass and branching ratios, we would drastically reduce the number of free SUSY model parameters.

We have yet to find any supersymmetric particle, and much parameter space has been ruled out, especially in simplified models.
However, there is still a large parameter space of more complicated models to be probed.
The exclusive decay channels will be more extensively probed by the increasing luminosity provided by the LHC in the coming decade.
However, a higher energy collider may provide the most promise for the discovery of supersymmetry if it exists.
