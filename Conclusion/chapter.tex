%This is the conclusion of the dissertation

\chapter[Conclusion][Conclusion]{Conclusion} %Your conclusion isn't a numbered chapter, so we use the asterisk here.

This thesis presented a search for supersymmetry in hadronic final states.
The dataset was near the highest integrated luminosity in history, and the proton-proton collisions had the highest center-of-mass energy every produced in a laboratory.

The search detailed in this thesis is the first to use Recursive Jigsaw Reconstruction.
This technique is the conceptual successor to the razor technique, and shows promise.
It compares favorably with previous techniques, and provides the ability to understand the underlying decay structure in additional detail.
As no excess was observed, we set model-dependent and model-independent limits in the simplified models of sparticle pair production consider.
It is useful to consider more broadly what has been learned by this analysis and dozens of other null searches for new physics at both ATLAS and CMS.
We discuss some stray thoughts for the future of searches for supersymmetry and new physics.

The assumption of $R$-parity is at the heart of a large number of LHC SUSY searches.
$R$-parity can not be too badly broken, as the proton is very stable, as discussed in \Cref{ch:intro,ch:susy}.
However, there is no particularly good reason to assume that all the $R$-parity violating (RPV) couplings are zero.
Any individual RPV coupling can be nonzero, while still avoiding the proton decay shown in \Cref{fig:proton_decay}.
With strong motivation from supersymmetry, the imposition of $R$-parity has two other effects.

$R$-parity conservation leads to a dark matter candidate.
Indeed, this candidate can be a WIMP, and this lucky coincidence is often known as the ``WIMP miracle'' \cite{darkMatterPrimer}.
However, it is possible that this miracle is simply a red herring.
The dark matter could be of a different nature than a weakly interacting massive particle, even assuming we discover supersymmetry with an appropriate LSP.
Additionally, the WIMPS could be real, but not coincide with the LSP from supersymmetry.
As evidence for dark matter is the best experimental motivation for supersymmetry, contemplation of these scenarios does not inspire confidence.

$R$-parity conservation makes searches for supersymmetry significantly easier.
In SUSY searches where $R$-parity is conserved, \met or related variables are strong discriminators against the dominant QCD background.
If $R$-parity is violated, the LSP will decay via SM particles, which can be measured by our experiments.
RPV searches do not have these discriminators against the most complicated background.
In order to more completely cover the phase space of $R$-parity violating supersymmetry, much more robust ways of modeling and understanding QCD backgrounds is a must.

Simplified models provide a useful tool to understand the reach of supersymmetric searches~\cite{whitePresusy}.
However, they can also lead us astray, as we must make ad-hoc assumptions which are not well-motivated.
Although they are not covered directly in this thesis, searches for supersymmetric tops are particularly affected by the branching ratio assumptions.
As both stops and tops have a variety of decay modes, the assumptions can drastically affect the final limits.
In future searches, there must be additional focus on understanding the simplified models inside of the larger space of the MSSM and more complicated supersymmetric models.

The space of supersymmetric models is \textit{very} large.
Even in the MSSM, we have 120 free parameters, which we have barely begun to explore.
Viewing the landscape from before Run-1, it is easy to see why the strategies detailed here became commonplace.
We expected to find some sort of new physics, which would help explain the hierarchy problem.
If we even discover one sparticle, with its associated mass and branching ratios, we would drastically reduce the number of free SUSY model parameters.

From our current point of view, this seems na{\"i}ve and overoptimistic.
We have yet to find any supersymmetric particle, and much parameter space has been ruled out, especially in simplified models.
However, we should not yet despair, as there is much more phase space to be explored, especially when we consider more complicated SUSY models.
Supersymmetric models will just have to be a little more fun.