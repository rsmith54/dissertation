%The following command starts your introduction. 

\chapter{Introduction}

%\addcontentsline{toc}{chapter}{Introduction} %This command puts your introduction in your table of contents even though we have used the asterisk in the \chapter command above.

Particle physics has been a remarkably successful field of scientific inquiry.
The ability to precisely predict the properties of a exceedingly wide range of physical phenomenom, from the cosmic microwave background (cite planck) to the anomalous magnetic moment of the muon (cite paper on this) is truly amazing.

The theory that has allowed this range of predictions has been the Standard Model of particle physics (SM) as developed by Gell-Mann, guy and guy.
This quantum field theory (QFT) contains a tiny number of particles, whose interactions describe phenomenom up to at least the \TeV\xspace scale.
These particles are manifestations of the fields of the Standard Model, after application of the Higgs Mechanism.
This 

Despite its impressive range of described phenomenom, the Standard Model has some theoretical and experimental deficiencies.
The SM contains 26 free parameters. \footnotemark \footnotetext{This is the Standard Model corrected for neutrino masses. These parameters are the fermion masses (6 leptons, 6 quarks), CKM and PMNS mixing angles (8 angles, 2 CP-violating phases), W/Z/Higgs masses (3) , the Higgs field expectation value, and the couplings of the strong, weak, and electromagnetic forces (3 $\alpha_{force}$ ) .}

This is a unique time in the history of particle physics.
After decades of waiting, the Large Hadron Collider began collisions in 2010, with all the accompanying fanfare and excitement.
The earliest searches failed to discover any new physical phenomenom; in 2011 and 2012, there was still no evidence of new physics, although statistical fluctuations continued to entice the community at 

Only a few short years ago, when the Large Hadron Collider was turned on for the first time, there was considerable excitement about the potential to observe new physical phenomenom.
%Today, some would say we are at a deadend, but we could also view this unique point in spacetime as a crossroads, wit
After 20 \ifb collected at $\sqrt{s} = $ 8 \TeV , there were some tantalizing hints, yet there was no smoking guns available.


This thesis documents one of the many searches performed at 13 \TeV by the ATLAS experiment.
In particular, this is a search for decays t

Particle physics 

Talk about SM in a paragraph

Talk about SUSY in a paragraph

Talk about RJR reconstruction in a paragraph