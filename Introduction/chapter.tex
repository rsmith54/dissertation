%The following command starts your introduction.

\chapter{Introduction}
\label{ch:intro}
%\addcontentsline{toc}{chapter}{Introduction} %This command puts your introduction in your table of contents even though we have used the asterisk in the \chapter command above.

Particle physics is a remarkably successful field of scientific inquiry.
The ability to precisely predict the properties of an exceedingly wide range of physical phenomena, such as the description of the cosmic microwave background~\cite{Perdereau:2016akt,Aghanim:2016sns}, the understanding of the anomalous magnetic dipole moment of the electron~\cite{Schwinger:1948iu, Laporta:1996mq}, and the measurement of the number of weakly-interacting neutrino flavors~\cite{ALEPH:2005ab} is truly amazing.

The theory that has allowed this range of predictions is the \textit{Standard Model} of particle physics (SM).
The Standard Model combines the electroweak theory of Glashow, Weinberg, and Salam~\cite{Glashow:1961tr, Weinberg:1967tq,  Salam:1968rm} with the theory of the strong interactions, as first envisioned by Gell-Mann and Zweig~\cite{GellMann:1964nj, Zweig:1964jf}.
This quantum field theory (QFT) contains a number of particles, whose interactions describe phenomena up to the \TeV\xspace scale.
These particles are manifestations of the fields of the Standard Model, after application of the Higgs Mechanism.
The particle content of the SM consists only of six quarks, six leptons, four gauge bosons, and a scalar Higgs boson.

The Standard Model has some theoretical and experimental deficiencies.
The SM contains 26 free parameters\footnotemark.
\footnotetext{This is the Standard Model corrected to include neutrino masses.
 These parameters are the fermion masses (6 leptons, 6 quarks), CKM and PMNS mixing angles (8 angles, 2 CP-violating phases), W/Z/Higgs masses (3), the Higgs field expectation value, and the couplings of the strong, weak, and electromagnetic forces (3 $\alpha_{force}$ ) .}
We would like to understand these free parameters in terms of a more fundamental theory.

The major theoretical concern of the Standard Model, as it pertains to this thesis, is the \textit{hierarchy problem}~\cite {Weinberg:1975gm,Weinberg:1979bn, Gildener:1976ai, Susskind:1978ms, susyPrimer}.
The light mass of the Higgs boson (125 \GeV)~\cite{HIGG-2012-27, CMS-HIG-12-028} should be quadratically dependent on the scale of UV physics, due to the quantum corrections from high-energy physics processes.
The most perplexing experimental issue is the existence of \textit{dark matter}, as demonstrated by galactic rotation curves~\cite{Rubin:1970zza, Roberts:1970zza, Rubin:1980zd, Rubin:1985ze, Bosma:1981zz, Persic:1995ru, darkMatterPrimer}.
This data has shown there exists additional matter which has not yet been observed interacting with the particles of the Standard Model.
There is no particle in the SM which can act as a candidate for dark matter.

Both of these major issues, as well as numerous others, can be solved by the introduction of \textit{supersymmetry} (SUSY)~\cite{Miyazawa:1966mfa, Gervais:1971xj, Gervais:1971ji, Golfand:1971iw, Neveu:1971rx, Neveu:1971iv, Volkov:1973ix,  Wess:1973kz, Salam:1974ig, Ferrara:1974ac, Wess:1974tw, susyPrimer,Lykken:1996xt,archilSUSYLectures}.
In supersymmetric theories, each SM particles has a so-called \textit{superpartner}, or sparticle partner, differing from given SM particle by $1/2$ in spin.
These theories solve the hierarchy problem, since the quantum corrections induced from the superpartners exactly cancel those induced by the SM particles.
In addition, these theories are usually constructed assuming $R-$parity, which can be thought of as the ``charge'' of supersymmetry, with SM particles having $R=1$ and sparticles having $R=-1$.
In collider experiments, since the incoming SM particles have total $R=1$, the resulting sparticles are produced in pairs.
This produces a rich phenomenology, which is characterized by significant hadronic activity and large missing transverse energy (\met), which provide significant discrimination against SM backgrounds~\cite{Farrar:1978xj}.

Despite the power of searches for supersymmetry where \met is a primary discriminating variable, there has been significant interest in the use of other variables to discriminate against SM backgrounds.
These include searches employing variables such as $\alpha_{T}$, $ M_{T,2}$, and the razor variables ($M_R, R^2$)~\cite{SUSY-2014-05, SUSY-2014-06, SUSY-2014-07, CMS-SUS-12-005, CMS-SUS-11-024, CMS-SUS-12-005, CMS-SUS-10-003, CMS-SUS-11-003, CMS-SUS-12-002,CMS-SUS-13-019, CMS-SUS-15-003, SUSY-2011-22}.
In this thesis, we will present the first search for supersymmetry using Recursive Jigsaw Reconstruction (RJR)~\cite{Jackson:2016mfb,ATLAS-CONF-2016-078}.
RJR can be considered the conceptual successor of the razor variables.
We impose a particular final state ``decay tree'' on an events, which roughly corresponds to a simplified Feynman diagram in decays containing weakly-interacting particles.
We account for the missing degrees of freedom associated with weakly-interacting particles by a series of simplifying assumptions, which allow us to calculate our variables of interest at each step in the decay tree.
This allows an unprecedented understanding of the internal structure of the decay and additional variables to reject Standard Model backgrounds.

This thesis describes a search for the superpartners of the gluon and quarks, the gluino and squarks, in final states with zero leptons, with 13.3 \ifb~of data using the ATLAS detector.
We organize the thesis as follows.
The theoretical foundations of the Standard Model and supersymmetry are described in Chapters 2 and 3.
The Large Hadron Collider and the ATLAS detector are presented in Chapters 4 and 5.
The reconstruction of physics objects is presented in Chapter 6.
Chapter 7 provides a detailed description of Recursive Jigsaw Reconstruction and a description of the variables used for the particular search presented in this thesis.
Chapter 8 presents the details of the analysis, including details of the dataset, object reconstruction, and selections used.
In Chapter 9, the final results are presented; since there is no evidence for a supersymmetric signal in the analysis, we present model-independent limits on the new physics cross-sections and the final exclusion curves in simplified supersymmetric models.
