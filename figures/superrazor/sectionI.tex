
\section{Introduction \label{sec:intro}}

Searches with the ATLAS and CMS detectors at the Large Hadron Collider have already placed strong lower bounds on the mass of pair-produced strongly-interacting gluinos or degenerate squarks decaying into final states with missing transverse energy \cite{Aad:2012naa,Chatrchyan:2012uea,Chatrchyan:2013fk,Aad:2013wta,CMS-PAS-SUS-13-004}. 
A  determination of the role of supersymmetry in electroweak symmetry breaking requires a much broader campaign of searches, many of which are already underway. %cites?
Some of these searches present special challenges at a hadron machine, even when they involve the pair production of relatively light superpartners. Examples include light stops whose decays closely resemble those of top quarks \cite{Carena:2008mj,Bi:2011ha,
Bai:2012gs,Alves:2012ft,Han:2012fw,Bhattacherjee:2012mz,Carena:2013iba,
Delgado:2012eu,Dutta:2012kx,Evans:2012bf,Kilic:2012kw,Buckley:2013lpa,Bai:2013ema}, a variety of models with compressed spectra, $R$-parity violating models \cite{Barbier:2004ez,Csaki:2011ge,Berger:2013sir,Evans:2012bf,CMS:yut,CMS:swa}, and relatively long-lived superpartners with displaced decays \cite{Aad:2013txa,Chatrchyan:2012jwg}. %cites for all of these

Of particular importance to this program is the direct electroweak production of charginos, neutralinos, and sleptons at the LHC. Relatively light charginos and neutralinos have a possible connection to weakly-interacting dark matter in supersymmetry models with conserved $R$-parity. Light sleptons are motivated by the measured value of the anomalous magnetic moment $(g-2)$ of the muon \cite{Jegerlehner:2009ry,Miller:2007kk}, providing a thermal annihilation cross section for bino-like neutralino dark matter \cite{Buckley:2013sca}, and the possibility that the branching fraction of the newly-discovered Higgs boson into two photons is enhanced over the Standard Model prediction \cite{Carena:2012gp}. 
Charginos, neutralinos, and sleptons could also appear in cascade decays of heavier colored superpartners, but this prospect merely emphasizes the importance of being able to produce these lighter superpartners directly.

We will focus on electroweak pair production of charged particles that decay to charged leptons and a stable (or long-lived) neutral particle, appearing in the detector only as missing transverse energy ($\vec{E}^\text{miss}_T$). The decay to leptons can occur either directly or through the leptonic decay of a $W$ boson. We will consider two canonical examples: sleptons of the first or second generation ($\tilde{e}^-\tilde{e}^+$ or $\tilde{\mu}^-\tilde{\mu}^+$) with 100\% branching into leptons and the lightest supersymmetric particle (LSP) neutralino, and charginos ($\tilde{\chi}^+\tilde{\chi}^-$) decaying through an on- or off-shell $W$ boson and the neutralino LSP. In the latter case, we require the $W$ to decay leptonically. In both cases, we set all other superpartner masses heavy, including the other charginos and neutralinos. Though our study is performed assuming a supersymmetric model, it can easily be generalized to other scenarios that contain similar particles with the same broad characteristics. The pair production of tau partners ({\it e.g.}~staus) has different backgrounds and will be considered in a later work.

Searches at LEP have already set lower bounds on the masses of new charged particles, ranging between 90 and 105 GeV assuming supersymmetric-like cross sections \cite{Beringer:1900zz}. The ATLAS and CMS collaborations have performed model-independent dilepton searches for both the slepton and chargino pair production scenarios we consider in this paper. ATLAS, using 20.3 fb$^{-1}$ of integrated luminosity at 8 TeV places an upper bound of 300 GeV on left-handed sleptons (assuming massless neutralinos), and an upper limit of 450~GeV on charginos assuming a 100\% branching ratio to leptons and neutralinos \cite{:2012gg,ATLAS-CONF-2013-049}. CMS places a 300~GeV bound on pair production of degenerate selectrons and smuons using 19.5~fb$^{-1}$ at 8 TeV, and 550~GeV on chargino pair production decaying to neutralinos with 100\% branching ratio \cite{CMS-PAS-SUS-12-022,CMS-PAS-SUS-13-006}. Both experiments \cite{CMS-PAS-SUS-12-022,CMS-PAS-SUS-13-006,Aad:2012ku,Aad:2012hba,ATLAS-CONF-2013-036,ATLAS-CONF-2013-035} have also performed multilepton searches for production of heavier chargino/neutralino pairs (such as $\tilde{\chi}^0_2 \tilde{\chi}^0_2$ or $\tilde{\chi}^0_2 \tilde{\chi}^\pm_1$), followed by cascades of the form $\tilde{\chi}^0_2 \to W^\pm \tilde{\chi}^\mp_1 \to \ell^\pm \ell^\mp \nu \bar{\nu} \tilde{\chi}^0_1$ to obtain three or more leptons in the final state.

%Additionally, some channels in the search for Higgs decaying to two $W$ bosons requires dileptons in the %final state. This analysis can be repurposed for searches in the pair production channels of interest here. A %theoretical study of the 7 TeV, 1.7~fb$^{-1}$ data set indicated that the experimental collaborations would %be sensitive to chargino pairs up 190 GeV (for massless neutralinos), assuming 100\% branching ratio into %leptons \cite{Lisanti:2011cj}.

We propose several techniques that can increase the sensitivity of the LHC experiments to electroweak pair production in the dilepton channel, using the data currently available from the completed 8 TeV run. Starting from the CMS razor variables \cite{Rogan:2010kb,Chatrchyan:2011ek} (see Refs.~\cite{Fox:2012ee,CMS:2012dwa,Chatrchyan:2012gq} for applications), 
we develop an improved version that more accurately approximates the production frame and center-of-mass (CM) energy scale of the pair production event, compared to the original razor formulation. This ``super-razor'' results in a set of mass variables, $\sqrt{\hat{s}_R}$ and $M_\Delta^R$ that contain information about the mass differences involved in the pair production and subsequent decay, allowing for discrimination between signal and background. In addition, the derivation of these mass variables involves constructing the approximate boost to the pair production frame, followed by a boost to an approximation of the decay frame. From this boost direction and the momenta of the visible particles, we construct angular variables $\Delta\phi_R^{\beta}$ and $|\cos\theta_{R+1}|$ that also distinguish between the signal events and background. Using these super-razor variables, we develop a new set of selection criteria and apply a multi-dimensional shape analysis to maximize the sensitivity to signal over the dominant backgrounds (primarily $W^-W^+$ and Drell-Yan + jets).  Shape analyses have been implemented by experimental groups \cite{Rogan:2010kb,CMS:2012dwa} and have been used in theoretical proposals for new searches \cite{Alves:2012ft,Fox:2012ee}. As we show through direct comparison to ATLAS- and CMS-like searches, this technique is promising in difficult channels.

In the next section we review the construction of the standard razor variables, followed by a derivation of the improved super-razor and the associated angular variables of interest. The background and signal simulations are described in Section~\ref{sec:simulation}, along with comparisons to the alternative searches by the ATLAS and CMS experiments that employed the kinematic variables $M_{T2}$ \cite{Lester:1999tx,Barr:2003rg} and $M_{CT\perp}$ \cite{Matchev:2009ad,Tovey:2008ui}. The shape analysis techniques and statistical tools are described in Section~\ref{sec:shape}. Our expected exclusion limits for 20~fb$^{-1}$ of integrated luminosity at 8 TeV are presented in Section~\ref{sec:conclusion}.

